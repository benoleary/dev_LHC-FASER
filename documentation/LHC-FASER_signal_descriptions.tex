\documentclass[article]
\pdfoutput=1
%\usepackage[utf8x]{inputenc}
\usepackage{graphicx}
\usepackage{graphics}
\usepackage[mathscr]{eucal}
\usepackage{amssymb}
\usepackage{amsmath}
\usepackage{xspace}
\usepackage{listings}
\usepackage{color}

\newcommand{\eg}[0]{\textit{e.g.}\xspace}
\newcommand{\ie}[0]{\textit{e.g.}\xspace}
\newcommand{\sec}[1]{section~\ref{#1}\xspace}
\newcommand{\faser}[0]{\texttt{LHC-FASER}\xspace}
\newcommand{\cpp}[0]{\texttt{C$++$}\xspace}
\newcommand{\lf}[1]{\texttt{LHC_FASER::#1}}
\newcommand{\BOL}[1]{\color{blue}{\textbf{BOL:}#1}}
\newcommand{\JL}[1]{\color{green}{\textbf{JL:}#1}}
\newcommand{\CR}[1]{\color{red}{\textbf{CR:}#1}}

%opening
% ==============================================================================
\title{Description Of The Signals Implemented In \faser}


\author{
Ben O'Leary${}^{1,a}$ \\
${}^{1}$Institut f{\"{u}}r Theoretische Physik und Astrophysik,
 Universit{\"{a}}t W{\"{u}}rzburg, \\
${}^{a}$Email: \email{benjamin.oleary@gmail.com} \\
}
% 
\abstract{
In this document the various signals that are implemented in \faser
 are described. The various approximations are also described, and possible
 improvements are identified.
 }

\begin{document}
\maketitle
\tableofcontents
% ==============================================================================
\section{Introduction}
\label{sec:intro}

The \cpp code \faser reads in a spectrum for the Minimal Supersymmetric Standard
 Model (MSSM) from a file in the SUSY Les Houches Accords (SLHA) format and
 returns approximate event rates for various signals. The code and accompanying
 lookup tables were created by Michael Kr{\"{a}}mer, Jonas Lindert, Ben O'Leary,
 and Carsten Robens.

\section{\faser Flow}
\label{sec:flow}

There are two stages in the intended workflow of \faser: the setup stage, and
 the calculation of rates for each point.

\subsection{Setup stage}
\label{subsec:setupflow}

\begin{itemize}

\item[1:] The \texttt{LHC-FASER.hpp} header file is included.

\item[2:] An \lf{lhcFaser} object is constructed.

\item[3:] The required signals are added by the
          \lf{lhcFaser::addSignal( std::string const )} function. The given
          string is interpretted by the \lf{lhcFaser} object as one of the
          signals described in \sec{sec:signals}.
          This function returns a pointer to the \lf{signalHandler} object which
          calculates the event rate for this particular signal.
          In adding the signal, certain objects are constructed and lookup grids
          are loaded:

  \begin{itemize}

  \item[a:] If the cross-section grids for the production of the colored
            sparticles at the signal's energy have not yet been loaded, they are
            loaded.

  \item[b:] If the objects responsible for working out the acceptances for the
            various Standard Model (SM) particles that are produced by the
            cascade decays of the colored sparticles at the signal's energy have
            not yet been constructed, they are constructed.

  \item[c:] If the jet and Missing Transverse Energy (MET) cut acceptance grids
            for the signal's energy have not yet been loaded, they are loaded.

  \end{itemize}

          Signals share grids, and only one instance of any grid is loaded into
          memory within a given \lf{lhcFaser} object.

\end{itemize}


\subsection{Workflow for a new point}
\label{subsec:pointflow}

Once the \lf{signalHandler} objects have been constructed, they are used to
 calculate rates for different MSSM spectra as given in SLHA files. However,
 since \faser is not responsible for the generation of the SLHA files, the
 \lf{lhcFaser} object needs to be informed when to read in a new SLHA file.

\begin{itemize}

\item[1:] The \lf{lhcFaser} object is told to read in the SLHA file with its
          \lf{lhcFaser::updateForNewSlha( std::string const )} function. The
          string provides the name of the SLHA file. It does not matter if the
          name of the SLHA file is different for the new point or not, though
          the function may be used without an argument, and in this case the
          last name used is used again.
          When either version of this function is called, all the internal
          objects that are used in the calculation of the rates erase their
          cached values and prepare to re-calculate their values if required for
          this new point.

\item[2:] The \lf{signalHandler} pointers are then used to call their
          \lf{signalHandler::getValue()} functions. These functions collect
          various cached values from internal objects and combine them to obtain
          a final approximate event rate. Each cached value is re-calculated the
          first time it is sought after the \lf{lhcFaser} object performs its
          \lf{lhcFaser::updateForNewSlha( std::string const )} function.
          The flow of calculating the event rate is follows:

  \begin{itemize}

  \item[a:] Each pair of colored sparticles that can be directly produced is
            iterated through, and the (cached) interpolated cross-section from
            the appropriate grid is used to determine whether this channel is
            worth calculating fully: if the cross-section is too low, the
            subsequent cascades are not calculated.

  \item[b:] If cross-section for a channel is significant, all the (cached)
            possible yet non-negligible cascades for each of the sparticles are
            considered.
            This is to say that as the cascades are built from following direct
            decays of sparticles, at each stage of the cascade, any decays of
            the last sparticle in the cascade which result in a total branching
            ratio that is below a threshold are ignored when building the next
            stage of the cascade.

  \item[c:] If branching ratios of the pair of cascades combined with the
            cross-section of the channel, \ie the partial cross-section into
            the final state defined by the cascades, is not below a threshold,
            the (cached) acceptances for the appropriate numbers of leptons and
            jets from the cascades are combined with the (cached) interpolated
            value of the acceptance for the jet plus MET cut to obtain a final
            value to be returned.

  \end{itemize}

\item[3:] Other signals from the same \lf{lhcFaser} object benefit from using
          the cached values which are calculated for the new point when
          requested by the first \lf{signalHandler} pointer's
          \lf{signalHandler::getValue()} function. However, some cached values
          may only be relevant to some signals, and they are only be calculated
          when first required (in the context of a single SLHA file's data).

\end{itemize}


\section{\faser Signals}
\label{sec:signals}

The \faser signals are essentially a combination of which jet$+$MET cut set to
 use with which combination of leptons to use.


\subsection{Jet$+$MET cut sets}
\label{subsec:jetPlusMetSets}

\begin{itemize}

\item[No cut:] No jet or MET cuts are set. The total cross-section is only
               reduced by the lepton cuts.

\item[CMS 2-jet:] The signal is multiplied by the acceptance for two hard jets
                  with a cut on ${\alpha}_{T}$ as described by \BOL{Carsten,
                  what's the reference?}. The leading jet has to have
                  $p_{T} > $ \BOL{100?} GeV, the other $p_{T} > $ \BOL{80?} GeV.
                  The MET comes indirectly from ${\alpha}_{T} > $ \BOL{0.55?}.

\item[Atlas 3-jet:] The signal is multiplied by the acceptance for three hard
                    jets plus MET with a cut on $M_{\text{eff}}$ as described by
                    \BOL{Carsten, what's the reference?}. The leading jet has to
                    have $p_{T} > $ \BOL{100?} GeV, the others
                    $p_{T} > $ \BOL{40?} GeV. The MET must be $> $ \BOL{500?}
                    GeV. Additionally, the $M_{\text{eff}}$ must be
                    $> $ \BOL{500?} GeV. There are also cuts on the angular
                    separation of the jets.

\item[Atlas 4-jet:] As above for the Atlas 3-jet signal, but now \em{four} hard
                    jets are required, again with
                    $p_{T} > $ \BOL{100, 40, 40, 40?} GeV.

\end{itemize}


\subsection{Lepton cut sets}
\label{subsec:leptonSets}

Any combination of any number of any charge of either electrons or muons may be
 specified, though only a single acceptance cut and a single vetoing cut may be
 specified per signal. The acceptance and vetoing cuts are to say that for a
 signal requiring $n$ positrons for example, only events with exactly
 $n$ positrons with $p_{T} >$ acceptance cut \em{and} exactly zero positrons
 with $p_{T} <$ acceptance cut but $>$ vetoing cut, and \em{any} number of
 positrons with $p_{T} <$ vetoing cut are counted. For example, if the
 acceptance cut is 20 GeV and the vetoing cut is 10 GeV, and exactly 1 lepton
 of either charge and either flavor is required, an event with 1 ${\mu}^{+}$
 with $p_{T} = 25$ GeV and no other leptons will count; an event with 1 $e^{-}$
 with $p_{T} = 122$ GeV and 18 other leptons all with $p_{T} < 10$ GeV will
 count; an event with 1 ${\mu}^{+}$ with $p_{T} = 15$ GeV and no other leptons
 will not count; an event with 1 ${\mu}^{-}$ with $p_{T} = 32$ GeV,
 1 ${\mu}^{+}$ with $p_{T} = 12$ GeV and no other leptons will not count.

However, the signals are restricted to a finite number of combinations of
 jet$+$MET cuts with lepton cuts, so the current lepton cuts used are:
\begin{itemize}

\item[0:] Exactly zero leptons may pass the vetoing cut.

\item[1:] Exactly one (${\mu}^{\pm}, e^{\pm}$) must pass the acceptance cut
          and exactly zero further leptons may pass the vetoing cut.

\item[OSSF-OSDF:] Exactly two leptons, both of the opposite electric
                  charge \em{and} of the same flavor, must pass the acceptance
                  cut and exactly zero further leptons may pass the vetoing cut.

\item[same-sign dilepton:] Exactly two leptons, both of the same electric
                           charge, must pass the acceptance cut and exactly zero
                           further leptons may pass the vetoing cut.

\end{itemize}


\section{Approximations used}
\label{sect:approximations}


\section{Conclusions}
\label{sect:conclusions}

\begin{appendix}

\section{Improvements still to come}
\label{app:improvements}


\end{appendix}

\bibliography{LHCFASERbib}



\end{document}
